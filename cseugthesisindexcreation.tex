\chapter{Index Creation}
Here is an example of index creation:

Albert Einstein\index{Einstein} was a German-born\index{Einstein!German-born} theoretical physicist who developed the theory of relativity, one of the two pillars of modern physics\index{Modern physics}. His work is also known for its influence on the philosophy of science. He is best known to the general public for his mass–energy equivalence formula $E = mc^2$\index{Einstein!Mass-energy equivalance formula }, which has been dubbed the world's most famous equation. He received the 1921 Nobel Prize\index{Nobel Prize} in Physics for his services to theoretical physics, and especially for his discovery of the law of the photoelectric effect, a pivotal step in the development of quantum theory.


Sir Isaac Newton\index{Newton}  was an English mathematician, physicist, astronomer, theologian, and author (described in his own day as a natural philosopher) who is widely recognised as one of the most influential scientists of all time, and a key figure in the scientific revolution. His book Philosophiæ Naturalis Principia Mathematica (Mathematical Principles of Natural Philosophy)\index{Newton!Philosophiæ Naturalis Principia Mathematica}, first published in 1687\index{1687|see {Philosophiæ Naturalis Principia Mathematica}}, laid the foundations of classical mechanics. Newton also made seminal contributions to optics, and shares credit with Gottfried Wilhelm Leibniz\index{Gottfried Wilhelm Leibniz} for developing the infinitesimal calculus.
